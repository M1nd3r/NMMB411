%08/10/2021
\begin{definition}[Gap SVP$_\gamma$]
(rozhodovací verze SVP$_\gamma$)\\
$L \subseteq (\Real ^n, +)$ mříž, úplná ( hodnost = $n$), víme, že $\lambda_1(L) \leq 1 $ nebo $\lambda_1(L) \geq \gamma(n)$. Máme rozhodnout, který případ nastává.
\end{definition}

\textbf{Learning with errors}: Odvozuje se od BDD$_\gamma$ (bounded distance decoding) 

\begin{definition}[BDD$_\gamma$ - bounded distance decoding]
$L \subseteq (\Real^n, +), v\in \Real^n$. Víme: $dist(v,L) < \frac{\lambda_1(L)}{2\gamma(n)}$. Chceme najít vektor $l\in L$, který tuto nerovnost dokazuje, tedy splňuje $$\|v-l\| < \frac{\lambda_1(L)}{2 \gamma(n)}$$.
\end{definition}

\begin{note}
$l_1 \neq l_2 \in L$, $\|l_1-l_2\| \geq \lambda_1(L)$

Pak  $ |\{u \in \Real^n : \|u-v\|<\frac{\lambda_1(L)}{2} \cap L\}| \leq 1$
\end{note}

\begin{definition}[$i$-té postupné minimum]
$L \subseteq (\Real^n, +)$ mříž hodnosti d. Pro $i \in \{1, \dots, d\}$ definujeme \emph{i-té postupné minimum} $\lambda_i(L) = \min\{r \in \Real: L \text{obsahuje i LN vektorů normy } \leq r\}$
\end{definition}

\begin{definition}[SIVP$_\gamma$ - short independent vectors problem]
Dána $L \subseteq (\Real^n, +) $ úplná. Chceme nalézt $S = \{s_1, \ldots, s_n\} \subseteq L$ lineárně nezávislé tak, aby $\|s_i\|\leq\gamma(n)\cdot\lambda_n(L)$.
\end{definition}

\begin{definition}[SIS$_{n,q,s,m}$ - short integer soultion]
Nechť $q \in \N$. Volíme náhodně $a_1, a_2, \dots, a_m \in \Z^n_q$. $A:= (a_1|a_2|\cdots|a_m),\ n \times m$ nad $\Z_q$. Chceme najít $0 \neq t\in\Z^m \  \|z\|\leq\beta, Az\equiv0\ (\modulo q)$
\end{definition}

\begin{note}
$L = \{n \in \Z^m: An \equiv 0\ (\modulo q)\}$ je celočíselná mříž obsahující $q:\Z^m$ ($q$-ární mříž)
\end{note}

\begin{example}
$2^m > q^n (m > n \cdot \log (q))$. Vezmeme $f_A: \{0,1\}^m \rightarrow \Z^n_q \ f_A(n):=An\ \modulo q$

$\exists u_1 \neq u_2 \in \{0,1\}^m: f_A(u_1) = f_A(u_2)$

$z=u_1-u_2\in\{0,1\}^m, O<\|z\|\leq\sqrt{m},\ Az\equiv0\ (\modulo q)$.

Potom $z$ řeší SIS$_{n,q,\sqrt{m},n}$.
\end{example}

\begin{theorem}[M. Ajatai, 1996]
Nechť $m=\poly(n), q\geq\beta\poly(n)$. Pokud existuje algoritmus řešící SIS$_{n,q,\beta,n}$ s nezanedbatelnou pravděpodobností, pak existuje srovnatelně efektivní algoritmus, který řeší SIVP$_\gamma$ s nezanedbatelnou pravděpodobností pro všechny instance $n-$dimenzionálních mříží, kde $\gamma=\poly(n)\cdot\beta$.
\end{theorem}

\begin{example}
Nechť $2^m>q^n$, $\beta\geq\sqrt{m}$.
Díváme se na $\{f_A: A \in M_{n,m}(t_q)\}$ jako na množinu hashovacích funkcí, která má $q^n$ prvků. Hledáme v ní náhodnou kolizi (speciální případ SIS$_{n,q, \beta,m}$).

Důkaz obtížnosti SIVP$_\gamma$ pro odpovídající $\gamma$ povede k důkazu obtížností problému hledání kolizí.
\end{example}

\section{\texorpdfstring{Lineární algebra nad $\Z$}{Linearni algebra na Z}}
\begin{definition}[volná grupa]
konečně generovaná komutativní grupa $G$ je \emph{volná}, pokud $\exists b_1, b_2, \dots, b_d \in G$ takové, že $\forall g \in G\ \ \exists ! z_1, z_2, \dots, z_d \in \Z$ tak, aby $g = z_1 b_1 +  z_2b_2 + \dots + z_d b_d$. Množina $\{b_1,b_2, \dots, b_d\}$ se nazývá \emph{volná báze $G$}.
\end{definition}

\begin{note}
$G = O$ volná grupa s volnou bází $\emptyset$

$L \subseteq (\Real^n, +)$ mříž. Potom báze mříže je volná báze grupy $(L,+)$

$(\Z^n, +)$ Potom volná báze např. $\{e_1, e_2, \dots , e_n\}, e_1 = \begin{pmatrix}
1 \\
0 \\
\vdots \\
0
\end{pmatrix}, \dots,
e_n=\begin{pmatrix}
0 \\
\vdots \\
0 \\
1
\end{pmatrix}$
\end{note}


\begin{claim}
Konečně generovaná volná grupa je izomorfní $(\Z^n,+)$ pro nějaké $n\in\N$.
\end{claim}
\begin{proof}
$G$ s volnou bází is $\{b_1,+\ldots,b_d\}\ \varphi:\Z^d\rightarrow G$ $\begin{pmatrix}
z_1 \\
\vdots \\
z_d \\
\end{pmatrix}
\rightarrow \sum_{i=1}^d z_i b_i$ je izomorfizmus grup.
\end{proof}


\begin{claim}
$(\Z^{d_1}, +) \simeq (\Z^{d_2}, +) \Rightarrow d_1 = d_2$
\end{claim}

\begin{proof}
$\varphi: \Z^{d_1} \rightarrow \Z^{d_2} $

$\varphi/2\Z^{d_i}: 2\Z^{d_1} \rightarrow \Z^{d_2}$



tyto dvě věci implikují: $\Z^{d_1}/2\Z^{d_2}\simeq \Z^{d_2}/2\Z^{d_2}\implies 2^{d_1}=2^{d_2}\implies d_1=d_2.$
\end{proof}

\begin{consequence}
$\{b_1, \dots, b_d\}, \{b_1', \dots, b'_d\}$ volné báze komutativní volné grupy $G \Rightarrow d = d'$. ($G \simeq (\Z^d, +) \simeq (\Z^{d'}, +)$)
\end{consequence}

\begin{definition}[rank grupy]
Rankem volné komutativní grupy $G$ rozumíme počet prvků nějaké její volné báze.
\end{definition}

\begin{claim}
$\forall\varphi: (\Z^n, +) \rightarrow (\Z^m, +)$ hom. $\exists ! A\in M_{m,n}(\Z)$ tak, že $\varphi(u)=A\cdot u \forall u\in\Z^n$.
\end{claim}

\begin{proof}
Pro $i = 1, \dots , n$: $\varphi(e_i) =: a_i \in \Z^m$. Dále $A:= (a_1|a_2|\dots|a_n)$. Potom $Au = A \begin{pmatrix}
u_1 \\
\vdots \\
u_n
\end{pmatrix}
=\sum_{i=1}^n u_i a_i=\sum_{i=1}^n u_i \varphi(e_i)
= \varphi(\sum_{i=1}^nu_ie_i) = \varphi(\begin{pmatrix}
u_1 \\
\vdots \\
u_n
\end{pmatrix}) = \varphi(n)$.

Jednoznačnost: $\varphi(u)=A\cdot u\implies \varphi(e_i)=A e_i\implies \varphi(e_i)$ musí být $i$-tý sloupec matice $A$. 
\end{proof}

\subsection{Souřadnice}
Nechť $G$ je konečně generovaná volná komutativní grupa a $B = \{b_1, \dots, b_d\}$ je volná báze $G$.

Pro $g \in G$ je $[g]_B = \begin{pmatrix}
z_1 \\
\vdots \\
z_d
\end{pmatrix} \in \Z^d$, kde $g = \sum_{i = 1}^d z_ib_i$, souřadnice $g$ vzhledem k bázi $B$.

\begin{definition}[Matice homomorfismu]
Nechť $0 \neq G,H$ jsou konečně generované volné komutativní grupy, $B_G$ volná báze $G$, $B_H$ volná báze $H$.

$\varphi: G \rightarrow H$ homomorfismus $[\varphi]_{B_H}^{B_G}$ je matice $|B_H| \times |B_G|$ nad $\Z$ 
splňující $[\varphi]_{B_H}^{B_G}\cdot[g]_{B_G}=[\varphi(g)]_{B_H}$ pro každé $g\in G$
\end{definition}

Sestrojí se tak, že $[\varphi]_{B_H}^{B_G} = ([\varphi(b_1)]_{B_H}|[\varphi(b_2)]_{B_H}| \dots | [\varphi(b_d)]_{B_H})$, $B_G=\{b_1,\ldots,b_d\}$

\begin{claim}
$\varphi:G\rightarrow H,\psi:H\rightarrow K,\ G,H,K$ volné komutativní grupy,

$B_G, B_H, B_K$ jejich volné báze

$[\psi \circ \varphi]_{B_K}^{B_G} = [\psi]_{B_K}^{B_H} \cdot [\psi]_{B_H}^{B_G}$
\end{claim}
\begin{proof}
Stejný důkaz jako v lineární algebře
\end{proof}