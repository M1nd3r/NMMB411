%26/11/2021
\section{LLL-redukovaná báze mříže}

\begin{definition}
$b_1,\ldots,b_n\in\Real^n$ je \emph{LLL-redukovaná}, pokud
\begin{itemize}
    \item (R1) $|\mu_{i,j}|\leq 1/2$ $\forall 1\leq j<i\leq b$
    \item (R2) $\|b_i^*\|^2\geq(3/4-\mu_{i,i-1}^2\|b_{i-1}^*\|^2$ $\forall 1<i\leq n$
\end{itemize}
Kde $b_1^*,\ldots,b_n^*$ je G-S ortogonalizace $b_1,\ldots,b_n$, $\mu_{i,j}$ jsou koeficienty z G-S ortogonalizace $b_1,\ldots,b_n$, $b_1^*=b_1$, $b_i^*=b_i-\sum_{j=1}^{i-1}\mu_{i,j}b_j^*$, kde 
$\mu_{i,j}=\frac{b_i\cdot b_j^*}{\|b_j^*\|}$
\end{definition}

\begin{note}[A. K. Lenstra, H. W. Lenstra, L. G. Lovász (1982)]
Factory polynomials with rational coefficients $f\in\Z [x]$ primitivní:
\begin{itemize}
    \item $p\in \PrimesSet$
    \item faktorizace $f\!\mod p$ v $\Z_p[x]$
    \item Henselova "zdvihnutím" TODO rozklad $f\!\mod p^k$ v $\Z_{p^k}[x]$
    \item Kombinae faktorů $~$ až $2^{\deg f -1}$ kombinací
\end{itemize}
Nahradit kombinací faktorů hledáním dostatečně krátkého vektoru v mříži.
\end{note}
\begin{note}
Dále si rozebereme podmínku (R2)\newline
$b_i^*\perp b_{i-1}^*$\\
$\|b_i^*+\mu_{i,j}b_{i-1}^*\|=\|b_i^*\|^2+\mu_{i,i-1}^2\|b_{i-1}^*\|$\\
(R2) $\|b_i^*+\mu_{i,i-1}b_{i-1}\|^2\geq 3/4 \|b_{i-1}\|^2$\\

$b_i^*$ je kolmá projekce $b_i$ do $\langle b_1,\ldots, b_{i-1}\rangle^\perp$\\
$b_i=b_i^*+\sum$
$L$ mříž s bází $b_1,\ldots,b_n$ $b_1,b_n$ je LLL-redukovaná.\\
$L=\Z b_1+\cdots +\Z b_n$, $1\leq i\leq n$\\
$L':=\Z b_{i-1}^* + \Z(b_i^*+\mu_{i,i-1}b_{i-1}^*$ je kolmá projekce $\Z b_1+ \cdots + \Z b_n$ do $\langle b_1,\ldots, b_{i-1}\rangle^\perp$\\

TODO There are missing parts.
\end{note}

\begin{lemma}[25.1 Ve skriptech] 
Nechť $b_1,\ldots b_n$ je LLL-redukovaná báze $\Real^n$. Pak $\|b_i\|^2\leq 2^{j-1}\|b_j^*\|$ $\forall 1\leq i \leq j \leq n$
\end{lemma}

\begin{proof}
(R1) + (R2) $\implies$ $\|b_i^*\|^2\geq \frac{1}{2}\|b_{i-1}^*\|^2$, indukcí $\|b_i^*\|^2\geq \frac{1}{2^l}\|b_{i-l}\|^2$ $0\leq l <i$\\
$$\|b_i\|^2=\|b_i^*+\sum_{j=1}^{i-1} \mu_{i,j}b_j^*\|^2=\|b_i^*\|^2+\sum_{j=1}^{i-1}\mu_{i,j}^2\|b_j^*\|\overset{(R1)}{\leq}\|b_i^*\|^2+\frac{1}{4}\sum_{j=1}^{i-1}\|b_{i-j}^*\|^2$$
$$\leq\|b_i^*\|^2+\frac{1}{4}\sum_{j=1}^{i-1}2^j\|b_i^*\|^2=
(1+\frac{1}{4}\underbrace{\sum_{j=1}^{i-1})\|b_i^*\|^2}_{2^{i-2}}=
(2^{i-2}+\frac{1}{2})\|b_i^*\|^2\leq 2^{i-1}\|b_i^*\|^2$$\\

$\forall i\geq 1$ $2^{i-2}+\frac{1}{2}\leq 2^{i-1}\iff 2^{i-1}+1\leq 2^i$.\\

Máme $\|b_i\|^2\leq 2^{i-1}\|b_i^*\|^2$ a zároveň $\|b_i\|^2\leq 2^{j-i}\|b_j^*\|^2$. Celkem tedy $\|b_i\|^2\leq 2^{i-1}2^{j-i}\|b_j^*\|^2=2^{j-1}\|b_j^*\|^2$.
\end{proof} 


\begin{claim}[25.2] Nechť $b_1,\ldots,b_n$ je LLL-redukovná báze mříže $L\subseteq (\Real^N,+)$. Pak $d(L)\leq\|b_1\|\cdots\|b_n\| \leq 2^{\frac{n(n-1)}{4}}d(L)$, $\|b_1\|\leq2^{\frac{n(n-1)}{4}}\sqrt[n]{d(L)}$.
\end{claim}

\begin{proof}
$d(L)^2=\|b_1^*\|^2\cdots\|b_n^*\|^2$ (platí pro každo bázi), $\|b_i\|\geq \|b_i^*\|\implies d(L)\leq \|b_i\|\cdots\|b_n\|$\\
Lemma 25.1 pro $i=j$: $\|b_i\|^2\leq 2^{i-1}\|b_i^*\|^2$ $i=1,\ldots, n$\\
$\implies \|b_1\|^2\cdots\|b_n\|^2\leq 2^{\sum_{i=1}^{n}i-1}\|b_1^*\|^2\cdots\|b_n^*\|^2=2^{\frac{n(n-1)}{2}}d(L)^2$\\
$\implies \|b_1\|^2\cdots\|b_n\|^2\leq 2^{\frac{n(n-1)}{4}}d(L)$

Lemma 25.1 $\implies \|\|$
TODO konec důkazu
\end{proof}

\begin{claim}[25.3]
$b_1,\ldots,b_n$ LLL-redukovaná báze mříže $L\subseteq(\Real^n,+).\ \forall 0\neq v\in L: \|b_1\|\leq 2^{\frac{n-1}{2}}\|v\|$
\end{claim}

\begin{proof}
$v=z_1 b_1+\cdots+z_n b_n$, $z_1,\ldots,z_n\in\Z$\\
$v\neq 0 : \exists k\ z_k\neq0$, $z_{k+1}=z_{k+2}=\cdots=z_n=0$\\
$v=\sum_{i=1}^k z_i b_i=\sum_{i-1}^k z_i(b_i^*+\sum_{j=1}^{i-1}\mu_{i,j}b_j^*)=z_k b_k^* + \sum_{j=1}^{k-1}r_j b_j^*$\\
$\|v\|^2=\underbrace{z_k^2}_{z_k^2\geq 1} \|b_k^*\|^2+\sum_{j=1}^{k-1} r_j^2\|b_j^*\|^2\geq \|b_k^*\|^2\overset{25.1}{\geq} \frac{1}{2^{k-1}}\|b_1\|^2$\\
$2^{n-1}\|v\|^2\geq 2^{k-1}\|v\|^2\geq 2^{k-1}\|v\|^2\geq\|b_1\|^2
\implies 2^{\frac{n-1}{2}}\|v\|\geq\|b_1\|$

$r_j\in\Real$

\end{proof}

\begin{definition}[LLL algoritmus]
(základní verze)\\
VSTUP: $b_1,\ldots,b_n$ báze $L\subseteq(\Z^n,+)$\\
VÝSTUP: LLL-redukovaná báze $L$

\begin{enumerate}
    \item G-S ortogonalizace $b_1, \ldots ,b_n$. Spočteme $b_1^*,\ldots ,b_n^*$ a $\mu_{i,j}$ pro $1\leq j \leq i \leq n$
    \item for i=2 to n do:\begin{itemize}
        \item for j=i-1 downto 1 do: \begin{itemize}
            \item $x:=\lfloor\mu_{i,j}\rceil$ //celočíselné zaokrouhlení
            \item $b_i:=b_i-xb_j$
            \item $\mu_{i,j}:=\mu_{i,j}-x$ // $\mu_{i,j}\in \langle -1/2, 1/2 \rangle$
            \item for l=1 to j-1 do: $\mu_{i,l}:=\mu_{i,l}-x\mu_{j,l}$
        \end{itemize}
    \end{itemize}
    \item for i=2 to n do:\begin{itemize}
        \item if $\|b_i^*\|^2<(3/4 - \mu_{i,i-1}^2)\|b_{i-1}^*\|$ then \begin{itemize}
            \item prohoď hodnoty v $b_i$ a $b_{i-1}$
            \item GOTO 1
        \end{itemize}
    \end{itemize}
    \item return $b_1,\ldots,b_n$
\end{enumerate}
\end{definition}

\begin{note}
V průběhu algoritmu v proměnných $b_1,\ldots, b_n$ je vždy báze $L$.\\
$b_1,\ldots, b_n$ báze $L$, $i\neq j$, $z\in\Z$\\
$b_1,\ldots,b_{i-1},b_i-xb_j,b_{i+2},\ldots, b_n$ také báze $L$.\\

Po skončení kroku 2 jsou v proměnných $b_1^*,\ldots, b_n^*,\mu_{i,j}$ data z G-S ortogonalizace báze v proměnných $b_1,\ldots, b_n$ $\implies$ po skončení kroku 2 báze v proměnných $b_1,\ldots, b_n$  splňuje (R1). Pokud nevyskočíme z kroku 3, je splněna podmína (R2) 
\end{note}

\begin{claim}
Nechť $b_1,\ldots, b_n$; $c_1,\ldots, c_n$ jsou dvě báze $\Real^n$, $x\in\Real$\\
$1\leq j<i\leq n$, $c_l=b_l\ \forall l\neq i$ a zároveň $c_i=b_i-xb_j$\\
$b_1^*,\ldots, b_n^*$ G-S ortogonalizace $b_1,\ldots, b_n$\\
$c_1^*,\ldots, c_n^*$ G-S ortogonalizace $c_1,\ldots, c_n$\\
Pak $b_l^*=c_l^*\ \forall l\in\{1,\ldots,n\}$

\end{claim}
\begin{proof}
$c_l^*$ je ortogonální projekce $c_l$ do $\langle c_1, \ldots, c_{l-1} \rangle^\perp$\\
$b_l^*$ je ortogonální projekce $b_l$ do $\langle b_1, \ldots, b_{l-1} \rangle^\perp$\\
Celkem $\implies c_l^*=b_l^*\ \forall l\neq i$

$c_i=b_i-\underbrace{xb_j}_{\in\langle b_1,\ldots,b_{i-1}\rangle}$ ortogonální projekce $c_i$ do $\langle b_1, \ldots, b_{i-1} \rangle^\perp=b_i^*+0\implies c_i^*=b_i^*$.
\end{proof}

