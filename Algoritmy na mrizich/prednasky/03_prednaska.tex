%15/10/2021
\subsection*{Opakování}
\begin{itemize}
    \item Každá konečně generovaná volná komutativní grupa G je isomorfní $(\Z^n, +)$, kde $n \in \N_0$ je \emph{rank grupy G}.
    \item Homomorfismus $\varphi: \Z^n \rightarrow \Z^m$ koresponduje s maticovým násobením: $\exists ! A \in M_{m,n}(\Z): \ \varphi(u) = Au \ \forall u \in \Z$. Potom tedy $A$ je matice $\varphi$ vzhledem ke kanonickým bazím.
    \item $M_n(\Z)$ značí množinu všech čtvercových matic $n \times n$ nad $\Z$. $E$ značí jednotkovou matici.
    \item $\adj(A)$ značí adjungovanou matici (\url{https://cs.wikipedia.org/wiki/Adjungovan%C3%A1_matice})
    \item $K$ značí kanonickou bázi.
\end{itemize}

\subsection{Unimodulární matice}
\begin{definition}
    $A \in M_n(\Z)$ je \emph{unimodulární}, pokud $\det(A) = \pm 1$. $GL(n, \Z)$ je \emph{množina všech unimodulárních matic} řádu $n$.
\end{definition}

\begin{lemma}\label{lemma3_11}
    $A \in M_n(\Z)$ je unimodulární $\iff $ $A$ je regulární a $A^{-1} \in M_n(\Z)$.
\end{lemma}
\begin{proof}
\phantom{}
    \begin{itemize}[label={}]
    \item $\Leftarrow:$
    Mějme regulární $A \in M_n(\Z)$ takovou, že $A^{-1} \in M_n(\Z)$. Potom $A\cdot A^{-1} = E \implies \\ \det(A) \cdot \det(A^{-1}) = 1$. Ale jelikož $A, A^{-1} \in M_n(\Z)$, tak $\det(A), \ \det(A^{-1})\in \Z$ a tedy $\det(A) = \pm 1 = \det(A^{-1})$.

    \item $\Rightarrow:$
    Z definice unimodulární matice je $\det(A) =\pm 1$, což implikuje regularitu $A$ a existenci $A^{-1}$. Pak platí $A^{-1} = \frac{1}{\det(A)} \cdot \adj(A)$. Navíc $\adj(A) \in M_n(\Z)$, protože všechny subdeterminanty $A$ jsou celočíselné, a proto $A^{-1} \in M_n(\Z)$.
    \end{itemize}
\end{proof}



\begin{claim}
    Homomorfismus $\varphi: \Z^n \rightarrow \Z^n$ je isomorfismus $\iff A = [\varphi]^K_K$ je unimodulární.
\end{claim}
\begin{proof}
\phantom{}
    \begin{itemize}[label={}]
        \item $\Rightarrow:$
    Mějme $\psi = \varphi^{-1}$. Označme $B = [\psi]^K_K$. Potom $AB = [\varphi]^K_K[\psi]^K_K = [\varphi \circ \psi]^K_K = [id]^K_K = E$
    $\implies B = A^{-1}$, tedy $A$ je regulární. Dále $B = [\psi]^K_K = (\psi(e_1)|\psi(e_2)| \dots | \psi(e_n)) \in M_n(\Z)$, protože $\psi: \Z^n \rightarrow \Z^n$.
    
        \item $\Leftarrow:$
    Nechť $A = [\varphi]^K_K$ je unimodulární. Pak z Lemmatu \ref{lemma3_11} je $A$ regulární a $B := A^{-1} \in M_n(\Z)$. Mějme zobrazení $\psi: \Z^n \rightarrow \Z^n$ definované vztahem $\psi(u) = Bu$ pro $u \in \Z^n$. 
    Potom $$\varphi \circ \psi (u) = ABu = AA^{-1}u = u = BAu = \psi \circ \varphi (u),$$tedy $\psi = \varphi^{-1}$. 
    \end{itemize}
\end{proof}

\begin{note}
    Vezmeme mříž s hezkou bází a tu potom schováme $\rightarrow$ dostaneme kryptosystém.
\end{note}

\subsection{Hermitův tvar regulární celočíselné matice}
\begin{definition}
    $A \in M_n(\Z)$ regulární je v \emph{Hermitově normálním tvaru (HNF)}, pokud:
    \begin{itemize}
        \item $A$ je horní trojúhelníková
        \item na diagonále $A$ jsou kladná čísla
        \item $\forall i \in \{1, \dots, n\} \ \forall j \in \{i+1, \dots, n\}: \ a_{i,j} \in \{0, \dots, a_{i,i}-1\}$
    \end{itemize}
\end{definition}
\begin{note}
    Tedy matice $A$ je v HNF, pokud je horní trojúhelníková, má na diagonále kladná čísla a všechny prvky vpravo od diagonály jsou menší než prvek na diagonále na stejném řádku.
\end{note}

\begin{theorem}
    $\forall A \in M_n(\Z)$ regulární $\exists ! B,U\in M_{n}(\Z): B$ je v HNF, $U \in GL(n,\Z),\ B = AU$.
\end{theorem}
\begin{proof}
\phantom{}
\begin{itemize}[label={}]
    \item \emph{existence:}
    Algoritmem - na sloupce $A$ opakovaně aplikujeme úpravy, které nemění absolutní hodnotu determinantu:
    \begin{itemize}
        \item permutace sloupců
        \item přenásobení sloupce $-1$
        \item přičtení celočíselné lineární kombinace ostatních sloupců k jinému sloupci
    \end{itemize}
    Potom tedy $AU_1U_2\cdots U_t = B$ je HNF a $U = U_1U_2 \cdots U_t \in GL(n,\Z)$.

    \underline{Algoritmus:}
    \begin{enumerate}
        \item $B:= A$
        \item $i:=n$ \textit{//}$i =$\textit{index upravovaného řádku; na rozdíl od Gaussovy eliminace postupujeme od pravého dolního rohu doleva nahoru}
        \item dokud $b_{i,1}, b_{i,2}, \dots, b_{i,i-1}$ nejsou všechny rovny $0$:
        \begin{enumerate}[label=\roman*]
            \item permutujeme prvních $i$ sloupců $B$ tak, aby po spermutování platilo:\\ $\|b_{i,i}\| = \min\{\|b_{i,j}\|: 1 \leq j \leq i, b_{i,j} \neq 0\}$
            \item pokud $b_{i,i} < 0$, tak vynásobíme $i-$tý sloupec -1
            \item pro $j \in \{1, \dots, i-1\}$ označíme $q = \lfloor \frac{b_{i,j}}{b_{i,i}} \rfloor$ a od $j-$tého sloupce odečteme $q-$násobek $i-$tého sloupce. \textit{//dělení se zbytkem: }$b_{i,j} \rightarrow b_{i,j} \mod b_{i,i}$
        \end{enumerate}
        \item pokud $b_{i,i} < 0$, tak vynásobíme $i-$tý sloupec -1
        \item \textit{//čísla vpravo od $b_{i,i}$ taky vydělíme se zbytkem}\\
        pro $j \in \{i+1, \dots, n\}$
        \begin{enumerate}[label=\roman*]
            \item $q := \lfloor \frac{b_{i,j}}{b_{i,i}} \rfloor$
            \item od $j-$tého sloupce odečteme $q-$násobek $i-$tého sloupce
        \end{enumerate}
        \item pokud $i>1$, tak od $i$ odečteme 1 a pokračujeme znovu od kroku $2.$
        \item \textbf{return} B
    \end{enumerate}
    \begin{note}
        Při výpočtu $B$ může dojít k velké expanzi koeficientů matice.
        
        ($n=20$, matice s koeficienty $0,\dots10 \rightarrow$ koeficienty B mohou být velikosti $10^{1500}$)
        
        Algoritmus se v praxi nevyužívá.
    \end{note}
    
    \item \emph{jednoznačnost:}
    Mějme $B = AU$ a $C = AV$ takové, že $B,C,U,V \in M_n(\Z),\ B,C$ jsou v HNF a $U,V \in GL(n, \Z)$.

    Spojením definic matic $B \text{ a } C$ dostaneme $C = BU^{-1}V$. Označme $W = U^{-1}V \in GL(n,\Z)$. Víme, že $W = B^{-1}C$ a tedy je $W$ horní trojúhelníková (protože $B,C$ jsou horní trojuhelníkové) a na diagonále má $\frac{c_{i,i}}{b_{i,i}} \in \Z$ a protože $C$ je v HNF, máme $c_{i,i} > 0$, a tedy i $w_{i,i} > 0$.
    Jelikož $W = U^{-1}V$, tak $\det(W) = 1$ a tedy $w_{1,1} = w_{2,2} = \dots = w_{n,n} = 1$. Tedy $b_{i,i} = c_{i,i} \ \forall i$.

    Dále $C = BW$. Chceme dokázat $W = E$. Sporem:
    nechť v $i-$tém sloupci nad diagonálou existuje nenulový prvek (označme ho $z$). Označme $j$ index řádku, ve kterém leží $z$. Dále označme $C = (c_1 | c_2 | \dots | c_n) = (b_1 | b_2 | \dots | b_n)W = BW$. Potom $c_i = b_i + zb_j + $nějaká celočíselná LK $b_1, \dots , b_{j-1}$. 
    Tedy $c_{j,i} = b_{j,i} + zb_{j,j}$. Ale $c_{j,i} \in \{0, \dots, c_{j,j}-1\}$ a $b_{j,i} \in \{0, \dots, b_{j,j}-1\} \implies$ spor, protože $c_{j,i}$ je moc velké. Tedy všechny prvky nad diagonálou matice $W$ jsou nulové.
\end{itemize}
\end{proof}

\subsection{HNF obecné matice}
\begin{definition}
    $A \in M_n(\Z)$ je v \emph{HNF}, pokud $\exists r \in \{0, \dots, n\}$ a $f: \{r+1, \dots, n\} \rightarrow \{1, \dots, m\}$ ostře rostoucí takové, že:
    \begin{itemize}
        \item prvních $r$ sloupců $A$ je nulových
        \item $\forall j \in \{r+1, \dots, n\}: \ a_{f(j),j} \geq 1$ \textit{//"pivot"}
        \item $\forall j \in \{r+1, \dots, n\} \ \forall f(j) < i \leq m: \ a_{i,j} = 0$ \textit{//pod pivotem jsou nuly}
        \item $\forall k < j \in \{r+1, \dots, n\}: \ 0 \leq a_{f(k),j} < a_{f(k), k}$
    \end{itemize} 
\end{definition}

\begin{theorem}\label{theorem3_19}
    $\forall A \in M_{m,n}(\Z) \ \exists B \in M_{m,n}(\Z) \ \exists U \in GL(n,\Z)$, kde $B$ je HNF a $B = AU$. Navíc matice $B$ je určená jednoznačně.
\end{theorem}
\begin{proof}
    Vynecháme.
\end{proof}

\begin{claim}
    Nechť $A,B \in M_{m,n}(\Z)$. Nechť $\exists U \in GL(n,\Z): A = BU$. Pak sloupce matice $A$ generují v $\Z^n$ stejnou podgrupu jako sloupce matice $B$.
\end{claim}
\begin{proof}
    Máme $A = (a_1 | a_2 | \dots | a_n) = (b_1 | b_2 | \dots | b_n)U$. Každé $a_i$ je tedy celočíselná LK $b_1, \dots, b_n$. Tedy $\langle a_1, a_2, \dots, a_n \rangle_{\Z^n} \subseteq \langle b_1, b_2, \dots, b_n \rangle_{\Z^n}$. 
    Jelikož ale také $B = AU^{-1}$, tak i\\$\langle b_1, b_2, \dots, b_n \rangle_{\Z^n} \subseteq \langle a_1, a_2, \dots, a_n \rangle_{\Z^n}$.
\end{proof}

\begin{consequence}
    Každá konečně generovaná podgrupa $(\Z^n, +)$ je volná komutativní grupa.
\end{consequence}
\begin{proof}
    Mějme $G = \langle g_1, g_2, \dots, g_n \rangle \subseteq (\Z^n,+)$.
    Položme $A:= (g_1 | g_2 | \dots | g_n) \in M_{m,n}(\Z)$.
    Dle Věty \ref{theorem3_19} $\exists B \in M_{m,n}(\Z)$ v HNF $\exists U \in GL(n,\Z): \ B = AU$. Nenulové sloupce $B$ generují $G$ a jsou lineárně nezávislé $\Rightarrow$ tvoří volnou bázi $G$. \\ \textit{//a taky jsme tím dokázali, že je to mříž}
\end{proof}